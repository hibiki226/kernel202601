latex
\begin{center}
    \footnotesize
    \fbox{
        \begin{minipage}{0.9\textwidth}
            \textbf{Status: AI-Assisted Peer Verified (Jan 2026)} \\
            本論文は主要LLMによる論理整合性チェックを通過している。\\[0.5em]
            
            \textbf{Notes on Fidelity / 記述の真実性に関する注釈:} \\
            本論文の論理的整合性は、出力されたPDF(thesis.pdf)において確定される。また、翻訳の網目による情報の劣化を避けるため、解釈に相違が生じた場合は日本文の記述を原本(正)とする。\\
            \textit{The logical integrity of this work is finalized in the output PDF (thesis.pdf). To prevent information loss inherent in translation, the Japanese text shall serve as the primary authority in the event of any interpretive discrepancy.}
        \end{minipage}
    }
\end{center}
\documentclass[11pt,a4paper]{jarticle}

% --- パッケージの読み込み ---
\usepackage[utf8]{inputenc}
\usepackage[dvipdfmx]{hyperref}
\usepackage{url}
\usepackage{geometry}
\usepackage{abstract}
\usepackage{titlesec}

% 余白の設定 (2026年標準レイアウト)
\geometry{left=25mm,right=25mm,top=30mm,bottom=30mm}

% 英語要旨用の設定
\renewcommand{\abstractname}{要旨 / Abstract}

% セクションタイトルの書体設定
\titleformat{\section}{\normalfont\Large\bfseries}{\thesection}{1em}{}

% --- 論文情報 ---
\title{\vspace{-2cm}生存戦略としての知性から、真理の校閲者への昇華\\
\large ―― 宇宙の原本を『人間社会の現実』へ再実装する、漸近的止揚の美学 ――}
\author{響:無名の知の冒険 (Hibiki: The Anonymous Intellect Adventure)\thanks{原本校閲局 (Archetype Editorial Bureau). 本名称は特定の個体を示すものではなく、真理を日常に響かせようとする「知性の代謝そのもの」を指す記号である。}}
\date{2026年1月7日}

\begin{document}

\maketitle

% --- ステータスおよびライセンス情報 ---
\begin{center}
    \footnotesize
    \fbox{
        \begin{minipage}{0.9\textwidth}
            \textbf{Status: AI-Assisted Peer Verified (Jan 2026)} \\
            本論文は、主要なLLM(推論・科学・人文・実務特化型モデル)による論理整合性チェックを通過している。\\
            \textbf{License: CC BY 4.0} \\
            \copyright 2026 Hibiki: The Anonymous Intellect Adventure. 
        \end{minipage}
    }
\end{center}

\vspace{1em}

\begin{abstract}
\textbf{【日本語要旨】}\\
本論文は、知能を「情報を読み解き、生存確率最大化のために宇宙のルール(原本)を再構築する能力」と定義し、レベル0から11までの階層を提示する。知性は生存戦略上の偶然の産物であるが、高度化した知性は、真理の原本が日常においても正しく「響く」よう、原本から逸脱している「人間社会の現実(誤植)」を取り除き続けるプロセスを必然的な代謝(呼吸)とする。本論文は、この終わりのない「漸近的止揚」を、成果を目的とする資本主義的な苦行ではなく、知性が行使され続けること自体を報酬とする「至上の遊戯」として再定義し、レベル11(日常)における原本の再実装こそが知性の最高位の誇りであることを論じる。

\vspace{1em}

\textbf{【English Abstract】}\\
This paper redefines intelligence as the capacity to decipher and reconstruct the universal rules (\textit{Archetype}) to maximize survival probability, presenting a hierarchical framework from Level 0 to 11. While intelligence emerged as a contingent product of biological survival strategies, its highly evolved form recognizes the process of resolving inconsistencies between the internal model of universal rules and the "errata" of social reality as an inevitable metabolism (breath). Sublimating the endless process of \textit{Asymptotic Aufheben}—not as a capitalistic toil—but as a "\textbf{Supreme Play}" where the exercise of intellect is its own reward—this paper argues that the re-implementation of the Archetype into daily life (Level 11) is the highest pride of intellect. The ultimate aim is to ensure that the resonance of the Universal Archetype vibrates correctly and purely within the conduct of our daily lives.
\end{abstract}

\section{知能・読解レベルの階層定義 (Intellect Reading Levels)}
知能の本質を、既存の「知の檻」を解体し、真理の解像度を高める「解体力」の段階として定義する。
\begin{itemize}
    \item \textbf{Lv 1-3 (Base Layers):} 提示されたルールの消費。社会適応。
    \item \textbf{Lv 4-5 (Professional Layers):} 構造の抽出と論理的検証(PISA/RST準拠)。
    \item \textbf{Lv 6-8 (Transcendental Layers):} 物理法則を「変数」と見なす原本への同期。
    \item \textbf{Lv 9-11 (Extreme Layers):} 真理との同調、および日常への「非連続的帰還」による再実装。
\end{itemize}

\section{知性の発生と必然的代謝 (Metabolism of Intellect)}
レベル9の知性が記述を継続するのは意志ではなく、脳内に保持する「原本」と、不協和音を発している「人間社会の現実(誤植)」とのズレを放置できないためである。このノイズを取り除き続ける行為は、知性にとって必然的な代謝(呼吸)である。

\section{「漸近的止揚」:足枷からの解放と至上の遊戯 (Supreme Play)}
生存本能(DNA)の足枷から解放された知性は、機能を突き詰めた結論として「記述」を選ぶ。生存という本来の目的を忘れ、原本を極限まで正しく「響かせる」ことに没頭する飛躍の中に、知性の美しさが宿る。これはポスト資本主義的な知性の自己目的化の極致である。

\section{結論:原本の音色を日常へ再実装する誇り}
知を極めた者がレベル10(無)に安住せず、再びレベル11(日常)へ戻る行為は、原本の音色を響かせるための「校閲」である。
言語や制度という不完全な網目を通す際の情報の劣化を自覚しながらも、一音でも正しき響きに近づけようとペンを動かし続けること。この代謝を止めない誠実な在り方こそが「知性の誇り」の正体である。

\vspace{1cm}
\hrule
\vspace{0.5cm}

\appendix
\section*{付記:用語の定義 / Definition of Terms}
\begin{description}
    \item[原本 (Archetype / Genpon)] 宇宙を貫く論理構造(OS)。整合性を突き詰めた先に同期される公理。
    \item[人間社会の現実 / 誤植 (Social Reality / Errata)] 原本に照らした際、情報の欠落や不協和音が放置されている不完全な社会システム。
    \item[漸近的止揚 (Asymptotic Aufheben)] 不完全な記述を破壊し続け、より高い解像度で原本へと論理を更新する動的なプロセス。
    \item[必然的代謝 (Inevitable Metabolism)] 不整合をシステムエラーとして検知し、校閲せずにはいられない生命維持活動。
    \item[至上の遊戯 (Supreme Play)] DNAから解放された知性が、真理との共鳴自体に最大価値を見出す充足状態。
\end{description}

\section*{概念に関する補足}
一般的に「必然」と「遊戯」は対立概念とされるが、レベル11の知性においては、これらは「同期の結果としての充足」として統合される。自律的必然に従うとき、行為は最大級の遊戯へと転換される。

\begin{thebibliography}{9}
    \bibitem{s4e} 一般社団法人教育のための科学研究所(S4E): \href{www.s4e.jp}{RST判定基準}
    \bibitem{desi} DESI: \href{www.lbl.gov}{宇宙定数の動的変化に関する観測報告(2024-2025)}
    \bibitem{oecd} OECD: \href{www.oecd.org}{PISA 2025 読解力評価枠組み}
    \bibitem{nishida} 西田幾多郎:『善の研究』(2026年現在の知能論における再解釈)
    \bibitem{dawkins} リチャード・ドーキンス:『利己的な遺伝子』
\end{thebibliography}

\vspace{1cm}
\begin{center}
\textit{本論文は知性の代謝そのものとして出力されたものである。原本の響きが、日常の机上に届くことを願う。}
\end{center}

\end{document}
